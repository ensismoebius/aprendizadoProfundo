\documentclass[3p,times]{elsarticle}
\usepackage{placeins}
\usepackage{ecrc}
\usepackage{amsthm}
\usepackage[figuresright]{rotating}
\usepackage{graphics}
\usepackage{amssymb}
\usepackage{graphicx}
\usepackage{fancybox}
\usepackage{amsmath, mathtools}
\usepackage{picinpar}
\usepackage{colortbl}
\usepackage{wasysym}
\usepackage{txfonts}
\usepackage{pb-diagram}
\usepackage{relsize}
\usepackage{longtable}
\usepackage{hyperref}
	\hypersetup{
		hidelinks = true,
		bookmarks = true,
		breaklinks= true
	}
	
\urlstyle{same}
	
\usepackage{multicol}
\setlength{\columnseprule}{0.1pt} % Set the width of the vertical line

\usepackage{longtable}

\usepackage{csvsimple, booktabs}

\usepackage{tikz}
	\usetikzlibrary{calc}
	\usetikzlibrary{datavisualization}
	\usetikzlibrary{positioning}
	\usetikzlibrary{mindmap}
	\usetikzlibrary{decorations}
	\usetikzlibrary{shapes}
	\usetikzlibrary{decorations.pathreplacing}
	\usetikzlibrary{spy}
	\usetikzlibrary{backgrounds}
	\usetikzlibrary{snakes}
\usepackage{pgfplots}
\usepackage{pgfplotstable}
	\pgfplotsset{compat=newest}
	\usepgfplotslibrary{units}
%\usepackage{subfigure}
%\usepackage{algorithm}
%\usepackage{algorithmic}
\usepackage{verbatim}
\usepackage{wrapfig}
\usepackage{array}
\usepackage{calc}


	% Allow more floats whatever it is
	\usepackage{morefloats}
	\usepackage{subfig}
	\usepackage{float}
	
	% Enables code listing
	\usepackage{listings}
	

\definecolor{codegreen}{rgb}{0,0.6,0}
\definecolor{codegray}{rgb}{0.5,0.5,0.5}
\definecolor{codepurple}{rgb}{0.58,0,0.82}
\definecolor{backcolour}{rgb}{0.95,0.95,0.92}

\lstdefinestyle{mystyle}{
	backgroundcolor=\color{backcolour},   
	commentstyle=\color{codegreen},
	keywordstyle=\color{magenta},
	numberstyle=\tiny\color{codegray},
	stringstyle=\color{codepurple},
	basicstyle=\ttfamily\footnotesize,
	breakatwhitespace=false,         
	breaklines=true,                 
	captionpos=b,                    
	keepspaces=true,                 
	numbers=left,                    
	numbersep=5pt,                  
	showspaces=false,                
	showstringspaces=false,
	showtabs=false,                  
	tabsize=2
}

\lstset{style=mystyle}


\newcommand{\boxplot}[7]{
	\draw[line width=0.3mm,color=#7] let \n{boxxl}={#1-0.1}, \n{boxxr}={#1+0.1} in (axis cs:\n{boxxl},#3) rectangle (axis cs:\n{boxxr},#4); % draw the box
	
	\draw[line width=0.3mm, color=#7] let \n{boxxl}={#1-0.1}, \n{boxxr}={#1+0.1} in (axis cs:\n{boxxl},#2) -- (axis cs:\n{boxxr},#2); % median
	
	\draw[line width=0.3mm, color=#7] (axis cs:#1,#4) -- (axis cs:#1,#6); % bar up
	
	\draw[line width=0.3mm,color=#7] let \n{whiskerl}={#1-0.025}, \n{whiskerr}={#1+0.025} in (axis cs:\n{whiskerl},#6) -- (axis cs:\n{whiskerr},#6); % upper quartile
	
	\draw[line width=0.3mm,color=#7] (axis cs:#1,#3) -- (axis cs:#1,#5); % bar down
	
	\draw[line width=0.3mm,color=#7] let \n{whiskerl}={#1-0.025}, \n{whiskerr}={#1+0.025} in (axis cs:\n{whiskerl},#5) -- (axis cs:\n{whiskerr},#5); % lower quartile
}

\volume{-}
\firstpage{1}
\journalname{ELS}
\runauth{}
\jid{ELS}
\jnltitlelogo{Elsevier}