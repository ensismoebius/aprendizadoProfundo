\par Among the methods of Brain-Computer Interface (BCI), Electroencephalogram (EEG) stands out as the most cost-effective and simple system to implement. However, it does have some quirks, such as high sensitivity to electromagnetic interference and difficulty in capturing the signal due to suboptimal scalp placement of the electrodes.

\par To address these issues, numerous Neural Networks (NNs) have been developed to handle such data. Despite their effectiveness, conventional NNs still require a significant amount of computational power for operation and training. This work proposes an alternative approach to this problem by utilizing Spiking Neural Networks (SNNs), which hypothetically consume less power. The aim is to assess the viability of SNNs in this context.

\par The results indicate a very close similarity in performance, demonstrating that SNNs are indeed a promising option for future research endeavors.

\par The sources used in this work are available in \url{https://github.com/ensismoebius/deepLearnning})