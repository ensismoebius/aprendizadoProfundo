\section{Introduction}
\label{introduction}
	\par Standard Neural Networks (NNs), as defined here—\textit{multilayered, fully connected, with or without recurrent or convolutional layers}—require that all neurons are activated in both the forward and backward passes. This implies that every unit (neuron) in the network must process some data, leading to power consumption \textbf{CITATION HERE}.
	
	\par In an era where responsible use of power resources is mandatory, coupled with the growing interest in Brain-Computer Interface (BCI) devices, a new problem arises: How to create NN models that are power-efficient both in usage and training, enabling their deployment even on portable devices?

	\par In light of the considerations, the objective of this paper is to test if the current state of Spiking Neural Networks (SNNs) can be an effective method in the processing of Electroencephalogram (EEG) data.
	
	\par In this work \textbf{PUT THE NUMBER AND KINDS OF NEURAL NETWORKS CREATE HERE} and use the \textbf{PUT THE MEASUMENT USED HERE} 

	\par The experiments and results described hereafter, relevantly complemented with tables and graphics lead to a set of interesting conclusions from the point of view of digital signal processing and intelligent systems. Thus, this piece of work provides a relevant contribution, supporting future investigations.
	
	\par The remaining of this document is organized as follows: Section \ref{sec:revBibli} presents a short literature review, Section \ref{sec:propApproach} describes the proposed approach, Section \ref{sec:testsResults} details the tests and the results and, lastly, Section \ref{sec:conclusions} is dedicated to the conclusions that are followed by the references. 