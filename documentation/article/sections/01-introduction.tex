\section{Introduction}
	\label{introduction}
	\par Standard Neural Networks (NNs), as defined here —\textit{multilayered, fully connected, with or without recurrent or convolutional layers}— require that all neurons are activated in both the forward and backward passes. This implies that every unit (neuron) in the network must process some data, leading to power consumption \cite{10242251}.
	
	\par In an era where responsible use of power resources is mandatory, coupled with the growing interest in Brain-Computer Interface (BCI) devices, a new problem arises: How to create NN models that are power-efficient, enabling their deployment even on portable devices?

	\par The objective of this paper is to test if the current state of Spiking Neural Networks (SNNs) can be an effective method in the processing of Electroencephalogram (EEG) data.
	
	\par In this work a Spiking Neural Network was created and compared with a Convolutional Neural Network (CNN) both in accuracy and loss metrics. 

	\par \textbf{CONCLUSIONS}
	
	\par The remaining of this document is organized as follows: Section \ref{sec:revBibli} presents a literature review, Section \ref{sec:experiments} details the tests and the results and lastly, Section \ref{sec:conclusions} is dedicated to the conclusions that are followed by the references. 