\section{Experiments}
	\label{sec:experiments}
	
	\subsubsection{Database}
		\par The used  \textbf{database} \cite{PRG16} can be found at \href{https://sinc.unl.edu.ar/grants/brain-computer-interfaces/}{https://sinc.unl.edu.ar/grants/brain-computer-interfaces} and consists in a set of EEG data recorded with 6 electrodes during a imagined speech of the words in spanish: 
		"A", "E", "I", "O", "U", "Arriba", "Abajo", "Adelante", "Atrás", "Derecha", "Izquierda".\newline
		
		\par Each channel captured 4096 samples and has the following ids: 'F3','F4', 'C3', 'C4', 'P3', 'P4' according to 10-20 system \cite{ScienceOpenVid:5960cfa8-7fde-441c-8592-35fdb9841499}.
	
	\subsubsection{Neural networks}
		\par Two neural networks were tested in this work:
		\begin{enumerate}
			\item \label{itm:snn} \textbf{SNN}: full connected spiking neurons
			\item \label{itm:cnn} \textbf{CNN}: convolutional layer with full connected sigmoid neurons, which were used as a reference.
		\end{enumerate}
	
		\subsection{Full connected spiking neurons SNN}
		\begin{figure}[H]
			\centering
			\includegraphics[width=.9\linewidth]{images/architectureSNN}
			\caption{Spiking neural network architecture}
			\label{fig:architecturesnn}
		\end{figure}
	
			\par The SNN were trained according to the following parameters:
			\begin{itemize}
				\item \textbf{random seed}: 42
				\item \textbf{learning rate}: $2.10^{-4}$
				\item \textbf{test batch size}: 40\%
				\item \textbf{training batch size}: 60\%
				\item \textbf{number of epochs}: 3000
				\item \textbf{loss function}: Cross-entropy
				\item \textbf{optimizer}: Adam ($0.00001 \leq \beta \leq 0.999$)
				\item \textbf{insist}: 4 (amount of times the same data is presented to the network)
				\item \textbf{neuron decay rate}: 0.9 (the rate LIF neuron loses its voltage)
			\end{itemize}
	
			\input{listings/snnArchitecture.py}
	
	
		\subsection{Convolutional layer with full connected sigmoid neurons CNN}
	
		\begin{figure}[H]
			\centering
			\includegraphics[width=\linewidth]{images/architectureCNN}
			\caption{Convolutional neural network architecture}
			\label{fig:architecturecnn}
		\end{figure}
	
			\par The CNN were trained according to the following parameters:
			\begin{itemize}
				\item \textbf{random seed}: 42
				\item \textbf{learning rate}: $2.10^{-4}$
				\item \textbf{test batch size}: 40\%
				\item \textbf{training batch size}: 60\%
				\item \textbf{number of epochs}: 3000
				\item \textbf{loss function}: Cross-entropy
				\item \textbf{optimizer}: Adam ($0.00001 \leq \beta \leq 0.999$)
			\end{itemize}
			\input{listings/cnnArchitecture.py}
			\onecolumn
			
		\subsection{Results}
			\subsubsection{Results for CNN}
				\par The CNN exhibit some unstable behavior and had a poor convergence as seen in Figure \ref{fig:cnnacculoss}. Its stats are the following:
				\begin{itemize}
					\item Maximum Accuracy: 4.82\%
					\item Minimum Loss: 0.9468
					\item Average Accuracy: 1.23\%
					\item Average Loss: 3.328
				\end{itemize}
				
				\begin{figure}[H]
					\centering
					\includegraphics[width=\linewidth]{images/cnnAccuLoss}
					\caption{CNN stats over 3000 epochs}
					\label{fig:cnnacculoss}
				\end{figure}
		
			\subsubsection{Results for SNN}
				\par The SNN exhibit a more stable behavior and had a better convergence then CNN as can be seen in Figure \ref{fig:snnacculoss}. Its stats are the following:
				\begin{itemize}
					\item Maximum Accuracy: 100.00\%
					\item Minimum Loss: 0.9278
					\item Average Accuracy: 18.25\%
					\item Average Loss: 3.296
				\end{itemize}
				\begin{figure}[H]
					\centering
					\includegraphics[width=\linewidth]{images/snnAccuLoss}
					\caption{SNN stats over 3000 epochs}
					\label{fig:snnacculoss}
				\end{figure}
			

		
			\twocolumn
	
	
	
